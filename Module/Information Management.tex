%%%%%%%%%%%%%%%%%%%%%%%%%%%%%%%%%%%%%%%%%%%%%%%%%%%
%% LaTeX book template                           %%
%% Author:  Amber Jain (http://amberj.devio.us/) %%
%% License: ISC license                          %%
%%%%%%%%%%%%%%%%%%%%%%%%%%%%%%%%%%%%%%%%%%%%%%%%%%%

\documentclass[a4paper,11pt,oneside]{book}
\usepackage{modulestyle}

%%%%%%%%%%%%%%%%%%%%%%%%%%%%%%%%%%%%%%%%%%%%%%%%%%%%%%%%%
% Source: http://en.wikibooks.org/wiki/LaTeX/Hyperlinks %
%%%%%%%%%%%%%%%%%%%%%%%%%%%%%%%%%%%%%%%%%%%%%%%%%%%%%%%%%

%%%%%%%%%%%%%%%%%%%%%%%%%%%%%%%%%%%%%%%%%%%%%%%%%%%%%%%%%%%%%%%%%%%%%%%%%%%%%%%%
% 'dedication' environment: To add a dedication paragraph at the start of book %
% Source: http://www.tug.org/pipermail/texhax/2010-June/015184.html            %
%%%%%%%%%%%%%%%%%%%%%%%%%%%%%%%%%%%%%%%%%%%%%%%%%%%%%%%%%%%%%%%%%%%%%%%%%%%%%%%%
\newenvironment{dedication}
{
   \cleardoublepage
   \thispagestyle{empty}
   \vspace*{\stretch{1}}
   \hfill\begin{minipage}[t]{0.66\textwidth}
   \raggedright
}
{
   \end{minipage}
   \vspace*{\stretch{3}}
   \clearpage
}

%%%%%%%%%%%%%%%%%%%%%%%%%%%%%%%%%%%%%%%%%%%%%%%%
% Chapter quote at the start of chapter        %
% Source: http://tex.stackexchange.com/a/53380 %
%%%%%%%%%%%%%%%%%%%%%%%%%%%%%%%%%%%%%%%%%%%%%%%%
\makeatletter
\renewcommand{\@chapapp}{}% Not necessary...
\newenvironment{chapquote}[2][2em]
  {\setlength{\@tempdima}{#1}%
   \def\chapquote@author{#2}%
   \parshape 1 \@tempdima \dimexpr\textwidth-2\@tempdima\relax%
   \itshape}
  {\par\normalfont\hfill--\ \chapquote@author\hspace*{\@tempdima}\par\bigskip}
\makeatother

%%%%%%%%%%%%%%%%%%%%%%%%%%%%%%%%%%%%%%%%%%%%%%%%%%%
% First page of book which contains 'stuff' like: %
%  - Book title, subtitle                         %
%  - Book author name                             %
%%%%%%%%%%%%%%%%%%%%%%%%%%%%%%%%%%%%%%%%%%%%%%%%%%%

\newcommand{\BookTitle}{Information Management (MySQL)}
\newcommand{\BookTitleFootnote}{A course in the Bachelor of Science in Computer Science.}

\newcommand{\BookSubtitle}{A Study Guide for Students of Sorsogon State University - Bulan Campus}
\newcommand{\BookSubtitleFootnote}{This book is a study guide for students of
Sorsogon State University - Bulan Campus taking up the course Information Management.}

\newcommand{\BookAuthorFirstName}{Jarrian Vince}
\newcommand{\BookAuthorLastName}{Gojar}
\newcommand{\BookAuthorName}{Jarrian Vince G. Gojar}
\newcommand{\BookAuthorURL}{https://github.com/godkingjay}

% Book's title and subtitle
\title{\Huge \textbf{\BookTitle}  \footnote{\BookTitleFootnote} \\
\huge \BookSubtitle \footnote{\BookSubtitleFootnote}}

% Author
\author{\textsc{\BookAuthorName}\thanks{\url{\BookAuthorURL}}}

\begin{document}

\frontmatter
\maketitle

%%%%%%%%%%%%%%%%%%%%%%%%%%%%%%%%%%%%%%%%%%%%%%%%%%%%%%%%%%%%%%%
% Add a dedication paragraph to dedicate your book to someone %
%%%%%%%%%%%%%%%%%%%%%%%%%%%%%%%%%%%%%%%%%%%%%%%%%%%%%%%%%%%%%%%
\begin{dedication}
  Sorsogon State University - Bulan Campus
\end{dedication}

%%%%%%%%%%%%%%%%%%%%%%%%%%%%%%%%%%%%%%%%%%%%%%%%%%%%%%%%%%%%%%%%%%%%%%%%
% Auto-generated table of contents, list of figures and list of tables %
%%%%%%%%%%%%%%%%%%%%%%%%%%%%%%%%%%%%%%%%%%%%%%%%%%%%%%%%%%%%%%%%%%%%%%%%
\tableofcontents
\listoffigures
\listoftables
\lstlistoflistings

\mainmatter

%%%%%%%%%%%
% Preface %
%%%%%%%%%%%
\chapter*{Preface}
% A Quote all about Databases
\begin{chapquote}{Daniel Keys Moran}
  ``You can have data without information, but you cannot have information without data.''
\end{chapquote}

\noindent \BookAuthorName \\
\noindent \url{\BookAuthorURL}

%%%%%%%%%%%%%%%%%%%%%%%%%%%%%%%%%%%%
%%%%%~ NEW CHAPTER STARTS HERE %%%%%
%%%%%%%%%%%%%%%%%%%%%%%%%%%%%%%%%%%%

\chapter{Introduction to MySQL}

\section{What is MySQL?}

\textbf{MySQL} is an open-source relational database management system
(RDBMS). An RDBMS is a type of database management system (DBMS) that
stores data in a structured format, using rows and columns. It is a
software that is used to manage the creation, modification, and
maintenance of a database. MySQL is a popular choice for database
management in web applications and is used by many high-profile
websites, including Facebook, Twitter, and YouTube.

\section{MySQL Elements}

\subsection{Tables}

A \textbf{table} is a collection of related data stored in rows and columns.
Each row in a table represents a record, and each column represents a field.
Tables are the basic building blocks of a database and are used to store and
organize data in a structured format.

% Example of a table
\begin{table}[!h]
  \centering
  \begin{tabular}{|c|c|c|}
    \hline
    \textbf{id} & \textbf{name} & \textbf{age} \\
    \hline
    1           & Alice         & 25           \\
    2           & Bob           & 30           \\
    3           & Charlie       & 35           \\
    \hline
  \end{tabular}
  \caption{Example of a table}
  \label{table:example}
\end{table}

Table \ref{table:example} shows an example of a table with three columns:
\textbf{id}, \textbf{name}, and \textbf{age}. Each row in the table represents
a record with a unique \textbf{id} value, and each column represents a field in
the record.

\subsection{Columns}

A \textbf{column} is a vertical arrangement of data in a table. Each column
represents a field in the table and contains data of a specific type. Columns
are used to store different types of data, such as numbers, text, dates, and
more.

In the example table above, the columns are \textbf{id}, \textbf{name}, and
\textbf{age}. The \textbf{id} column stores unique identifiers for each record,
the \textbf{name} column stores the names of the individuals, and the
\textbf{age} column stores the ages of the individuals.

\subsection{Rows}

A \textbf{row} is a horizontal arrangement of data in a table. Each row
represents a record in the table and contains data for each column in the
table. Rows are used to store individual records in the table.

In the example table above, each row represents a record with data for the
\textbf{id}, \textbf{name}, and \textbf{age} columns. The first row contains
data for Alice, the second row contains data for Bob, and the third row
contains data for Charlie.

\subsection{Data Types}

A \textbf{data type} is a classification of data based on the type of values it
can hold. MySQL supports a wide range of data types, including numeric, string,
date, and time data types. Data types are used to specify the type of data that
can be stored in a column in a table.

\subsubsection{Numeric Data Types}

Numeric data types are used to store numeric values, such as integers,
decimals, and floating-point numbers. MySQL supports a variety of numeric data
types, including \textbf{INT}, \textbf{DECIMAL}, \textbf{FLOAT},
\textbf{DOUBLE}, \textbf{TINYINT}, \textbf{SMALLINT}, \textbf{MEDIUMINT}, and
\textbf{BIGINT}.

\subsubsection{Character and String Data Types}

Character and string data types are used to store text values, such as names,
addresses, and descriptions. MySQL supports a variety of character and string
data types, including \textbf{CHAR}, \textbf{VARCHAR}, and \textbf{TEXT}.

\subsubsection{Date and Time Data Types}

Date and time data types are used to store date and time values, such as
birthdates, appointment times, and event dates. MySQL supports a variety of
date and time data types, including \textbf{DATE}, \textbf{TIME},
\textbf{DATETIME}, and \textbf{TIMESTAMP}.

\subsubsection{Binary Data Types}

Binary data types are used to store binary data, such as images, audio files,
and video files. MySQL supports a variety of binary data types, including
\textbf{BLOB}, \textbf{MEDIUMBLOB}, and \textbf{LONGBLOB}.

\subsubsection{Boolean Data Type}

The \textbf{BOOLEAN} data type is used to store boolean values, such as true or
false. In MySQL, boolean values are represented as 1 for true and 0 for false.

\subsection{NULL Values}

A \textbf{NULL} value is a special value that represents the absence of a
value. NULL values are used to indicate that a column does not contain any
data. In MySQL, columns can be defined to allow NULL values, which means that
the column can contain NULL values in addition to other values.

\subsection{Comments}

Comments are used to add explanatory notes to the SQL code. Comments are
ignored by the MySQL server and are used to document the code for reference.
Comments can be added to SQL code using the \textbf{--} or \textbf{/* */}
syntax.

\begin{lstlisting}[language=SQL, caption={Example of a comment in SQL code}, label={lst:comment}]
-- This is a single-line comment

/*
  This is a multi-line comment
  that spans multiple lines
*/

SELECT * FROM users; -- This is a comment at the end of a line
\end{lstlisting}

Code \ref{lst:comment} shows an example of comments in SQL code. Comments can
be added to SQL code using the \textbf{--} syntax for single-line comments and
the \textbf{/* */} syntax for multi-line comments.

\chapter{Managing Databases}

\section{Creating a Database}

To create a new database in MySQL, you can use the \textbf{CREATE DATABASE}
statement followed by the name of the database. The following example creates a
new database named \textbf{database\_name}:

\begin{lstlisting}[language=SQL, caption={Creating a new database in MySQL}, label={lst:create-database}]
CREATE DATABASE database_name;
\end{lstlisting}

Code \ref{lst:create-database} shows an example of creating a new database in
MySQL using the \textbf{CREATE DATABASE} statement. The statement creates a new
database named \textbf{database\_name}. Once the database is created, you can
use it to store tables and data.

\section{Showing Databases}

To display a list of all databases in MySQL, you can use the \textbf{SHOW
  DATABASES} statement. The following example shows a list of all databases in
MySQL:

\begin{lstlisting}[language=SQL, caption={Showing all databases in MySQL}, label={lst:show-databases}]
SHOW DATABASES;
\end{lstlisting}

Code \ref{lst:show-databases} shows an example of displaying a list of all
databases in MySQL using the \textbf{SHOW DATABASES} statement. The statement
lists all databases that are currently available in the MySQL server.

\section{Selecting a Database}

To select a database in MySQL, you can use the \textbf{USE} statement followed
by the name of the database. The following example selects the
\textbf{database\_name} database:

\begin{lstlisting}[language=SQL, caption={Selecting a database in MySQL}, label={lst:select-database}]
USE database_name;
\end{lstlisting}

Code \ref{lst:select-database} shows an example of selecting a database in
MySQL using the \textbf{USE} statement. The statement selects the
\textbf{database\_name} database, which allows you to perform operations on the
tables and data in that database.

\section{Removing a Database}

To remove a database in MySQL, you can use the \textbf{DROP DATABASE} statement
followed by the name of the database. The following example removes the
\textbf{database\_name} database:

\begin{lstlisting}[language=SQL, caption={Removing a database in MySQL}, label={lst:drop-database}]
DROP DATABASE database_name;
\end{lstlisting}

Code \ref{lst:drop-database} shows an example of removing a database in MySQL
using the \textbf{DROP DATABASE} statement. The statement deletes the
\textbf{database\_name} database and all tables and data stored in that
database.

\noindent\textbf{Exercises}

\begin{lstlisting}[numbers=none]
1. Create a new database named "employees".
2. Display a list of all databases in MySQL.
3. Select the "employees" database.
4. Remove the "employees" database.
\end{lstlisting}

\chapter{Managing Tables}

\section{Creating a Table}

To create a new table in MySQL, you can use the \textbf{CREATE TABLE} statement
followed by the name of the table and a list of columns. The following example
creates a new table named \textbf{table\_name} with three columns: \textbf{id},
\textbf{name}, and \textbf{age}:

\begin{lstlisting}[language=SQL, caption={Creating a new table in MySQL}, label={lst:create-table}]
CREATE TABLE table_name (
  id INT PRIMARY KEY AUTO_INCREMENT,
  name VARCHAR(50),
  age INT
);
\end{lstlisting}

Code \ref{lst:create-table} shows an example of creating a new table in MySQL
using the \textbf{CREATE TABLE} statement. The statement creates a new table
named \textbf{table\_name} with three columns: \textbf{id}, \textbf{name}, and
\textbf{age}. Each column is defined with a data type (INT or VARCHAR) and a
maximum length (50 for VARCHAR).

\section{Showing Tables}

To display a list of all tables in a database in MySQL, you can use the
\textbf{SHOW TABLES} statement. The following example shows a list of all
tables in the current database:

\begin{lstlisting}[language=SQL, caption={Showing all tables in MySQL}, label={lst:show-tables}]
SHOW TABLES;
\end{lstlisting}

Code \ref{lst:show-tables} shows an example of displaying a list of all tables
in the current database in MySQL using the \textbf{SHOW TABLES} statement. The
statement lists all tables that are currently available in the database.

\noindent\textbf{Exercises}

\begin{lstlisting}[numbers=none]
1. Display a list of all tables in the current database.
\end{lstlisting}

\section{Describe Table}

To display the structure of a table in MySQL, you can use the \textbf{DESCRIBE}
statement followed by the name of the table. The following example shows the
structure of the \textbf{table\_name} table:

\begin{lstlisting}[language=SQL, caption={Describing a table in MySQL}, label={lst:describe-table}]
DESCRIBE table_name;
\end{lstlisting}

Code \ref{lst:describe-table} shows an example of displaying the structure of a
table in MySQL using the \textbf{DESCRIBE} statement. The statement shows the
columns, data types, and other properties of the \textbf{table\_name} table.

\section{Renaming a Table}

To rename a table in MySQL, you can use the \textbf{RENAME TABLE} statement
followed by the current name of the table and the new name of the table. The
following example renames the \textbf{old\_table} table to \textbf{new\_table}:

\begin{lstlisting}[language=SQL, caption={Renaming a table in MySQL}, label={lst:rename-table}]
RENAME TABLE old_table TO new_table;
\end{lstlisting}

Code \ref{lst:rename-table} shows an example of renaming a table in MySQL using
the \textbf{RENAME TABLE} statement. The statement renames the
\textbf{old\_table} table to \textbf{new\_table}.

\section{Dropping a Table}

To drop a table in MySQL, you can use the \textbf{DROP TABLE} statement
followed by the name of the table. The following example drops the
\textbf{table\_name} table:

\begin{lstlisting}[language=SQL, caption={Dropping a table in MySQL}, label={lst:drop-table}]
DROP TABLE table_name;
\end{lstlisting}

Code \ref{lst:drop-table} shows an example of dropping a table in MySQL using
the \textbf{DROP TABLE} statement. The statement deletes the
\textbf{table\_name} table and all data stored in that table.

\noindent\textbf{Exercises}

\begin{lstlisting}[numbers=none]
1. Create a database named "library".
2. Create a table named "books" with columns:
    - id (INT, PRIMARY KEY, AUTO_INCREMENT)
    - title (VARCHAR, 100)
    - author (VARCHAR, 50)
    - year (INT)
3. Create a table named "users" with columns:
    - id (INT, PRIMARY KEY, AUTO_INCREMENT)
    - name (VARCHAR, 50)
    - email (VARCHAR, 100)
    - is_staff (BOOLEAN)
4. Display the structure of the "books" table.
5. Rename the "users" table to "members".
6. Drop the "members" table.
\end{lstlisting}

\section{Temporary Tables}

Temporary tables are a special type of table that is created and destroyed
automatically by the MySQL server. Temporary tables are useful for storing
temporary data that is only needed for the duration of a session or a query.
Temporary tables are created using the \textbf{CREATE TEMPORARY TABLE}
statement and are automatically dropped when the session ends.

\begin{lstlisting}[language=SQL, caption={Creating a temporary table in MySQL}, label={lst:create-temp-table}]
CREATE TEMPORARY TABLE temp_table (
  id INT PRIMARY KEY AUTO_INCREMENT,
  name VARCHAR(50)
);
\end{lstlisting}

Code \ref{lst:create-temp-table} shows an example of creating a temporary table
in MySQL using the \textbf{CREATE TEMPORARY TABLE} statement. The statement
creates a temporary table named \textbf{temp\_table} with two columns:
\textbf{id} and \textbf{name}. The temporary table is automatically dropped
when the session ends.

\noindent\textbf{Exercises}

\begin{lstlisting}[numbers=none]
1. Create a temporary table named "temp_data" with columns:
    - id (INT, PRIMARY KEY, AUTO_INCREMENT)
    - value (VARCHAR, 50)
2. Display a list of all tables in the current database.
\end{lstlisting}

\section{Altering Tables (Adding Columns and Modifying Columns)}

To alter a table in MySQL, you can use the \textbf{ALTER TABLE} statement
followed by the name of the table and the type of alteration to perform. The
following example adds a new column named \textbf{email} to the
\textbf{table\_name} table:

\begin{lstlisting}[language=SQL, caption={Altering a table in MySQL}, label={lst:alter-table}]
ALTER TABLE table_name ADD COLUMN email VARCHAR(100);
\end{lstlisting}

Code \ref{lst:alter-table} shows an example of altering a table in MySQL using
the \textbf{ALTER TABLE} statement. The statement adds a new column named
\textbf{email} with a data type of VARCHAR(100) to the \textbf{table\_name}
table.

If multiple alterations need to be performed on a table, you can use the
\textbf{ALTER TABLE} statement multiple times to apply each alteration
separately.

\begin{lstlisting}[language=SQL, caption={Altering a table in MySQL (multiple alterations)}, label={lst:alter-table-multiple}]
ALTER TABLE table_name
  ADD COLUMN email VARCHAR(100),
  ADD COLUMN phone VARCHAR(20),
  MODIFY COLUMN age INT;
\end{lstlisting}

Code \ref{lst:alter-table-multiple} shows an example of altering a table in
MySQL with multiple alterations using the \textbf{ALTER TABLE} statement. The
statement adds two new columns named \textbf{email} and \textbf{phone} and
modifies the data type of the \textbf{age} column in the \textbf{table\_name}
table.

\noindent\textbf{Exercises}

\begin{lstlisting}[numbers=none]
  1. In the books table, add new columns named "genre" (VARCHAR, 50), "isbn" (VARCHAR, 20), and "price" (DECIMAL).
  2. In the books table, modify the data type of the "year" column to SMALLINT.
\end{lstlisting}

\subsection{Dropping Columns}

To drop a column from a table in MySQL, you can use the \textbf{ALTER TABLE}
statement followed by the name of the table and the column to drop. The
following example drops the \textbf{email} column from the \textbf{table\_name}
table:

\begin{lstlisting}[language=SQL, caption={Dropping a column from a table in MySQL}, label={lst:drop-column}]
ALTER TABLE table_name DROP COLUMN email;
\end{lstlisting}

Code \ref{lst:drop-column} shows an example of dropping a column from a table
in MySQL using the \textbf{ALTER TABLE} statement. The statement drops the
\textbf{email} column from the \textbf{table\_name} table.

\noindent\textbf{Exercises}

\begin{lstlisting}[numbers=none]
1. In the books table, drop the "price" column.
\end{lstlisting}

\subsection{Renaming Columns}

To rename a column in a table in MySQL, you can use the \textbf{ALTER TABLE}
statement followed by the name of the table and the new name for the column.
The following example renames the \textbf{old\_name} column to
\textbf{new\_name} in the \textbf{table\_name} table:

\begin{lstlisting}[language=SQL, caption={Renaming a column in a table in MySQL}, label={lst:rename-column}]
ALTER TABLE table_name CHANGE COLUMN old_name new_name VARCHAR(50);
\end{lstlisting}

Code \ref{lst:rename-column} shows an example of renaming a column in a table
in MySQL using the \textbf{ALTER TABLE} statement. The statement renames the
\textbf{old\_name} column to \textbf{new\_name} with a data type of VARCHAR(50)
in the \textbf{table\_name} table.

\noindent\textbf{Exercises}

\begin{lstlisting}[numbers=none]
1. In the books table, rename the "isbn" column to "isbn_number" (VARCHAR, 20).
\end{lstlisting}

\section{Adding Constraints}

Constraints are rules that are applied to columns in a table to enforce data
integrity and consistency. Constraints can be used to specify conditions that
must be met for data to be inserted or updated in a table. MySQL supports a
variety of constraints, including \textbf{PRIMARY KEY}, \textbf{UNIQUE},
\textbf{NOT NULL}, and \textbf{FOREIGN KEY}.

\subsection{Primary Key Constraint}

The \textbf{PRIMARY KEY} constraint is used to uniquely identify each record in
a table. A primary key column cannot contain NULL values and must have a unique
value for each record. To add a primary key constraint to a column in MySQL,
you can use the \textbf{PRIMARY KEY} keyword in the column definition.

\begin{lstlisting}[language=SQL, caption={Adding a primary key constraint to a column in MySQL}, label={lst:primary-key}]
CREATE TABLE table_name (
  id INT PRIMARY KEY,
  name VARCHAR(50)
);
\end{lstlisting}

Code \ref{lst:primary-key} shows an example of adding a primary key constraint
to a column in MySQL using the \textbf{PRIMARY KEY} keyword. The statement
creates a new table named \textbf{table\_name} with a primary key column named
\textbf{id}.

\subsection{Unique Constraint}

The \textbf{UNIQUE} constraint is used to ensure that all values in a column
are unique. A unique constraint allows NULL values, but only one NULL value is
allowed. To add a unique constraint to a column in MySQL, you can use the
\textbf{UNIQUE} keyword in the column definition.

\begin{lstlisting}[language=SQL, caption={Adding a unique constraint to a column in MySQL}, label={lst:unique}]
CREATE TABLE table_name (
  id INT PRIMARY KEY AUTO_INCREMENT,
  name VARCHAR(50),
  email VARCHAR(100) UNIQUE
);
\end{lstlisting}

Code \ref{lst:unique} shows an example of adding a unique constraint to a
column in MySQL using the \textbf{UNIQUE} keyword. The statement creates a new
table named \textbf{table\_name} with a unique column named \textbf{email}.

\subsection{Not Null Constraint}

The \textbf{NOT NULL} constraint is used to ensure that a column cannot contain
NULL values. A not null constraint requires that a column must have a value for
each record. To add a not null constraint to a column in MySQL, you can use the
\textbf{NOT NULL} keyword in the column definition.

\begin{lstlisting}[language=SQL, caption={Adding a not null constraint to a column in MySQL}, label={lst:not-null}]
CREATE TABLE table_name (
  id INT PRIMARY KEY AUTO_INCREMENT,
  name VARCHAR(50) NOT NULL,
  email VARCHAR(100)
);
\end{lstlisting}

Code \ref{lst:not-null} shows an example of adding a not null constraint to a
column in MySQL using the \textbf{NOT NULL} keyword. The statement creates a
new table named \textbf{table\_name} with a not null column named
\textbf{name}.

\subsection{Foreign Key Constraint}

The \textbf{FOREIGN KEY} constraint is used to establish a relationship between
two tables in a database. A foreign key constraint specifies that the values in
a column must match the values in another column in a different table. To add a
foreign key constraint to a column in MySQL, you can use the \textbf{FOREIGN
  KEY} keyword in the column definition.

\begin{lstlisting}[language=SQL, caption={Adding a foreign key constraint to a column in MySQL}, label={lst:foreign-key}]
CREATE TABLE table_name (
  id INT PRIMARY KEY AUTO_INCREMENT,
  user_id INT,
  FOREIGN KEY (user_id) REFERENCES users(id)
);
\end{lstlisting}

Code \ref{lst:foreign-key} shows an example of adding a foreign key constraint
to a column in MySQL using the \textbf{FOREIGN KEY} keyword. The statement
creates a new table named \textbf{table\_name} with a foreign key column named
\textbf{user\_id} that references the \textbf{id} column in the \textbf{users}
table.

\section{Altering Tables with Constraints}

To alter a table in MySQL with constraints, you can use the \textbf{ALTER
  TABLE} statement followed by the name of the table and the type of alteration
to perform. The following example adds a primary key constraint to the
\textbf{id} column in the \textbf{table\_name} table:

\begin{lstlisting}[language=SQL, caption={Adding a primary key constraint to a column in MySQL}, label={lst:alter-primary-key}]
ALTER TABLE table_name ADD PRIMARY KEY (id);
\end{lstlisting}

Code \ref{lst:alter-primary-key} shows an example of adding a primary key
constraint to a column in a table in MySQL using the \textbf{ALTER TABLE}
statement. The statement adds a primary key constraint to the \textbf{id}
column in the \textbf{table\_name} table.

% Adding a NOT NULL constraint
\begin{lstlisting}[language=SQL, caption={Adding a not null constraint to a column in MySQL}, label={lst:alter-not-null}]
ALTER TABLE table_name MODIFY COLUMN name VARCHAR(50) NOT NULL;
\end{lstlisting}

Code \ref{lst:alter-not-null} shows an example of adding a not null constraint
to a column in a table in MySQL using the \textbf{ALTER TABLE} statement. The
statement modifies the \textbf{name} column in the \textbf{table\_name} table
to add a not null constraint.

\noindent\textbf{Exercises}

\begin{lstlisting}[numbers=none]
1. In the books table, add a unique constraint to the "isbn_number" column.
2. In the books table, add a not null constraint to the "title" column.
\end{lstlisting}

\section{Truncating Tables}

To remove all records from a table in MySQL, you can use the \textbf{TRUNCATE
  TABLE} statement followed by the name of the table. The following example
removes all records from the \textbf{table\_name} table:

\begin{lstlisting}[language=SQL, caption={Truncating a table in MySQL}, label={lst:truncate-table}]
TRUNCATE TABLE table_name;
\end{lstlisting}

Code \ref{lst:truncate-table} shows an example of truncating a table in MySQL
using the \textbf{TRUNCATE TABLE} statement. The statement removes all records
from the \textbf{table\_name} table, but the table structure remains intact.

\chapter{Managing Data}

\section{Inserting Data}

To insert data into a table in MySQL, you can use the \textbf{INSERT INTO}
statement followed by the name of the table and a list of values. The following
example inserts a new record into the \textbf{table\_name} table:

\begin{lstlisting}[language=SQL, caption={Inserting data into a table in MySQL}, label={lst:insert-data}]
INSERT INTO table_name (name, age) VALUES ('Alice', 25);
\end{lstlisting}

Code \ref{lst:insert-data} shows an example of inserting data into a table in
MySQL using the \textbf{INSERT INTO} statement. The statement inserts a new
record with values for the \textbf{name} and \textbf{age} columns in the
\textbf{table\_name} table.

It is also possible to insert multiple records into a table using a single
\textbf{INSERT INTO} statement. To insert multiple records, you can specify
multiple sets of values separated by commas.

\begin{lstlisting}[language=SQL, caption={Inserting multiple records into a table in MySQL}, label={lst:insert-multiple}]
INSERT INTO table_name (name, age) VALUES
  ('Alice', 25),
  ('Bob', 30),
  ('Charlie', 35);
\end{lstlisting}

\noindent\textbf{Exercises}

\begin{lstlisting}[numbers=none]
1. Create a database named "library".
2. Create a table named "books" with columns:
    - id (INT, PRIMARY KEY, AUTO_INCREMENT)
    - title (VARCHAR, 100)
    - author (VARCHAR, 50)
    - year (INT)
    - genre (VARCHAR, 50)
    - isbn (VARCHAR, 20)
3. Insert the following records in the "books" table:
\end{lstlisting}

% Table of 6 records to insert
\begin{table}[!h]
  \centering
  \begin{tabular}{|c|c|c|c|c|c|}
    \hline
    \textbf{id} & \textbf{title}        & \textbf{author}     & \textbf{year} & \textbf{genre}  & \textbf{isbn} \\
    \hline
    % 1 Fiction
    1           & The Great Gatsby      & F. Scott Fitzgerald & 1925          & Fiction         & 9780743273565 \\
    % 2 Horror
    2           & The Shining           & Stephen King        & 1977          & Horror          & 9780307743657 \\
    % 3 Mystery
    3           & The Da Vinci Code     & Dan Brown           & 2003          & Mystery         & 9780307474278 \\
    % 4 Science Fiction
    4           & Dune                  & Frank Herbert       & 1965          & Science Fiction & 9780441172719 \\
    % 5 Fantasy
    5           & The Hobbit            & J.R.R. Tolkien      & 1937          & Fantasy         & 9780618260300 \\
    % 6 Romance
    6           & Pride and Prejudice   & Jane Austen         & 1813          & Romance         & 9780141439518 \\
    % 7 Fiction
    7           & To Kill a Mockingbird & Harper Lee          & 1960          & Fiction         & 9780061120084 \\
    \hline
  \end{tabular}
  \caption{Records to insert in the "books" table}
  \label{table:insert-records}
\end{table}

% INSERT INTO books (title, author, year, genre, isbn_number) VALUES
%   ('The Great Gatsby', 'F. Scott Fitzgerald', 1925, 'Fiction', 9780743273565),
%   ('The Shining', 'Stephen King', 1977, 'Horror', 9780307743657),
%   ('The Da Vinci Code', 'Dan Brown', 2003, 'Mystery', 9780307474278),
%   ('Dune', 'Frank Herbert', 1965, 'Science Fiction', 9780441172719),
%   ('The Hobbit', 'J.R.R. Tolkien', 1937, 'Fantasy', 9780618260300),
%   ('Pride and Prejudice', 'Jane Austen', 1813, 'Romance', 9780141439518)
%   ('To Kill a Mockingbird', 'Harper Lee', 1960, 'Fiction', 9780061120084);

\section{Selecting Data}

To select data from a table in MySQL, you can use the \textbf{SELECT} statement
followed by a list of columns or the wildcard character (*). The following
example selects all records from the \textbf{table\_name} table:

\begin{lstlisting}[language=SQL, caption={Selecting data from a table in MySQL}, label={lst:select-data}]
SELECT * FROM table_name;
\end{lstlisting}

Code \ref{lst:select-data} shows an example of selecting data from a table in
MySQL using the \textbf{SELECT} statement. The statement selects all records
from the \textbf{table\_name} table and displays the data in the result set.

% \section{Filtering Data}

% \subsection{WHERE Clause}

% To filter data in a table in MySQL, you can use the \textbf{WHERE} clause in
% the \textbf{SELECT} statement to specify a condition. The following example
% selects records from the \textbf{table\_name} table where the \textbf{id} value
% is \textbf{1}:

% \begin{lstlisting}[language=SQL, caption={Filtering data in a table in MySQL}, label={lst:filter-data}]
% SELECT * FROM table_name WHERE id = 1;
% \end{lstlisting}

% Code \ref{lst:filter-data} shows an example of filtering data in a table in
% MySQL using the \textbf{WHERE} clause in the \textbf{SELECT} statement. The
% statement selects records from the \textbf{table\_name} table where the
% \textbf{id} value is \textbf{1}. Only records that match the condition are
% included in the result set.

% Say for example you want to select all records from the \textbf{books} table
% where the \textbf{genre} is \textbf{Fiction}, you can use the following SQL
% query:

% \begin{lstlisting}[language=SQL, caption={Filtering data in a table in MySQL}, label={lst:filter-data-genre}]
% SELECT * FROM books WHERE genre = 'Fiction';
% \end{lstlisting}

% Code \ref{lst:filter-data-genre} shows an example of filtering data in a table
% in MySQL using the \textbf{WHERE} clause in the \textbf{SELECT} statement. The
% statement selects records from the \textbf{books} table where the
% \textbf{genre} is \textbf{Fiction}.

% \subsection{AND Operator}

% To filter data using multiple conditions in MySQL, you can use the \textbf{AND}
% operator in the \textbf{WHERE} clause to combine conditions. The following
% example selects records from the \textbf{table\_name} table where the
% \textbf{year} value is \textbf{2020} and the \textbf{genre} value is
% \textbf{Science Fiction}:

% \begin{lstlisting}[language=SQL, caption={Filtering data using the AND operator in MySQL}, label={lst:filter-data-and}]
% SELECT * FROM table_name WHERE year = 2020 AND genre = 'Science Fiction';
% \end{lstlisting}

% Code \ref{lst:filter-data-and} shows an example of filtering data using the
% \textbf{AND} operator in MySQL. The statement selects records from the
% \textbf{table\_name} table where the \textbf{year} value is \textbf{2020} and
% the \textbf{genre} value is \textbf{Science Fiction}.

% \subsection{OR Operator}

% To filter data using multiple conditions in MySQL, you can use the \textbf{OR}
% operator in the \textbf{WHERE} clause to combine conditions. The following
% example selects records from the \textbf{table\_name} table where the
% \textbf{year} value is \textbf{2020} or the \textbf{genre} value is
% \textbf{Science Fiction}:

% \begin{lstlisting}[language=SQL, caption={Filtering data using the OR operator in MySQL}, label={lst:filter-data-or}]
% SELECT * FROM table_name WHERE year = 2020 OR genre = 'Science Fiction';
% \end{lstlisting}

% Code \ref{lst:filter-data-or} shows an example of filtering data using the
% \textbf{OR} operator in MySQL. The statement selects records from the
% \textbf{table\_name} table where the \textbf{year} value is \textbf{2020} or
% the \textbf{genre} value is \textbf{Science Fiction}.

\section{Updating Data}

To update data in a table in MySQL, you can use the \textbf{UPDATE} statement
followed by the name of the table and a list of columns and values to update.
The following example updates the \textbf{name} column in the
\textbf{table\_name} table:

\begin{lstlisting}[language=SQL, caption={Updating data in a table in MySQL}, label={lst:update-data}]
UPDATE table_name SET name = 'Bob' WHERE id = 1;
\end{lstlisting}

Code \ref{lst:update-data} shows an example of updating data in a table in
MySQL using the \textbf{UPDATE} statement. The statement updates the
\textbf{name} column to \textbf{'Bob'} for the record with an \textbf{id} value
of \textbf{1} in the \textbf{table\_name} table.

\noindent\textbf{Exercises}

\begin{lstlisting}[numbers=none]
1. Update the "title" column in the "books" table where the "id" is 1 to "The Great Gatsby (Revised Edition)".
2. Update the "year" column in the "books" table where the "id" is 2 to 1977.
\end{lstlisting}

\section{Deleting Data}

To delete data from a table in MySQL, you can use the \textbf{DELETE FROM}
statement followed by the name of the table and a condition to filter the
records to delete. The following example deletes records from the
\textbf{table\_name} table:

\begin{lstlisting}[language=SQL, caption={Deleting data from a table in MySQL}, label={lst:delete-data}]
DELETE FROM table_name WHERE id = 1;
\end{lstlisting}

Code \ref{lst:delete-data} shows an example of deleting data from a table in
MySQL using the \textbf{DELETE FROM} statement. The statement deletes records
from the \textbf{table\_name} table where the \textbf{id} value is \textbf{1}.

\noindent\textbf{Exercises}

\begin{lstlisting}[numbers=none]
1. Delete the record from the "books" table where the genre is "Romance".
\end{lstlisting}

\chapter{References}

\begin{enumerate}[label={\textbf{\Alph*.}}]
  \item \textbf{Books}
        \begin{itemize}
          \item
        \end{itemize}
  \item \textbf{Other Sources}
        \begin{itemize}
          \item
        \end{itemize}
\end{enumerate}

\end{document}
