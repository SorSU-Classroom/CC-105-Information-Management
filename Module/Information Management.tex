%%%%%%%%%%%%%%%%%%%%%%%%%%%%%%%%%%%%%%%%%%%%%%%%%%%
%% LaTeX book template                           %%
%% Author:  Amber Jain (http://amberj.devio.us/) %%
%% License: ISC license                          %%
%%%%%%%%%%%%%%%%%%%%%%%%%%%%%%%%%%%%%%%%%%%%%%%%%%%

\documentclass[a4paper,11pt,oneside]{book}
\usepackage{modulestyle}

%%%%%%%%%%%%%%%%%%%%%%%%%%%%%%%%%%%%%%%%%%%%%%%%%%%%%%%%%
% Source: http://en.wikibooks.org/wiki/LaTeX/Hyperlinks %
%%%%%%%%%%%%%%%%%%%%%%%%%%%%%%%%%%%%%%%%%%%%%%%%%%%%%%%%%

%%%%%%%%%%%%%%%%%%%%%%%%%%%%%%%%%%%%%%%%%%%%%%%%%%%%%%%%%%%%%%%%%%%%%%%%%%%%%%%%
% 'dedication' environment: To add a dedication paragraph at the start of book %
% Source: http://www.tug.org/pipermail/texhax/2010-June/015184.html            %
%%%%%%%%%%%%%%%%%%%%%%%%%%%%%%%%%%%%%%%%%%%%%%%%%%%%%%%%%%%%%%%%%%%%%%%%%%%%%%%%
\newenvironment{dedication}
{
   \cleardoublepage
   \thispagestyle{empty}
   \vspace*{\stretch{1}}
   \hfill\begin{minipage}[t]{0.66\textwidth}
   \raggedright
}
{
   \end{minipage}
   \vspace*{\stretch{3}}
   \clearpage
}

%%%%%%%%%%%%%%%%%%%%%%%%%%%%%%%%%%%%%%%%%%%%%%%%
% Chapter quote at the start of chapter        %
% Source: http://tex.stackexchange.com/a/53380 %
%%%%%%%%%%%%%%%%%%%%%%%%%%%%%%%%%%%%%%%%%%%%%%%%
\makeatletter
\renewcommand{\@chapapp}{}% Not necessary...
\newenvironment{chapquote}[2][2em]
  {\setlength{\@tempdima}{#1}%
   \def\chapquote@author{#2}%
   \parshape 1 \@tempdima \dimexpr\textwidth-2\@tempdima\relax%
   \itshape}
  {\par\normalfont\hfill--\ \chapquote@author\hspace*{\@tempdima}\par\bigskip}
\makeatother

%%%%%%%%%%%%%%%%%%%%%%%%%%%%%%%%%%%%%%%%%%%%%%%%%%%
% First page of book which contains 'stuff' like: %
%  - Book title, subtitle                         %
%  - Book author name                             %
%%%%%%%%%%%%%%%%%%%%%%%%%%%%%%%%%%%%%%%%%%%%%%%%%%%

\newcommand{\BookTitle}{Information Management (MySQL)}
\newcommand{\BookTitleFootnote}{A course in the Bachelor of Science in Computer Science.}

\newcommand{\BookSubtitle}{A Study Guide for Students of Sorsogon State University - Bulan Campus}
\newcommand{\BookSubtitleFootnote}{This book is a study guide for students of
Sorsogon State University - Bulan Campus taking up the course Information Management.}

\newcommand{\BookAuthorFirstName}{Jarrian Vince}
\newcommand{\BookAuthorLastName}{Gojar}
\newcommand{\BookAuthorName}{Jarrian Vince G. Gojar}
\newcommand{\BookAuthorURL}{https://github.com/godkingjay}

% Book's title and subtitle
\title{\Huge \textbf{\BookTitle}  \footnote{\BookTitleFootnote} \\
\huge \BookSubtitle \footnote{\BookSubtitleFootnote}}

% Author
\author{\textsc{\BookAuthorName}\thanks{\url{\BookAuthorURL}}}

\begin{document}

\frontmatter
\maketitle

%%%%%%%%%%%%%%%%%%%%%%%%%%%%%%%%%%%%%%%%%%%%%%%%%%%%%%%%%%%%%%%
% Add a dedication paragraph to dedicate your book to someone %
%%%%%%%%%%%%%%%%%%%%%%%%%%%%%%%%%%%%%%%%%%%%%%%%%%%%%%%%%%%%%%%
\begin{dedication}
Sorsogon State University - Bulan Campus
\end{dedication}

%%%%%%%%%%%%%%%%%%%%%%%%%%%%%%%%%%%%%%%%%%%%%%%%%%%%%%%%%%%%%%%%%%%%%%%%
% Auto-generated table of contents, list of figures and list of tables %
%%%%%%%%%%%%%%%%%%%%%%%%%%%%%%%%%%%%%%%%%%%%%%%%%%%%%%%%%%%%%%%%%%%%%%%%
\tableofcontents
\listoffigures
\listoftables
\lstlistoflistings

\mainmatter

%%%%%%%%%%%
% Preface %
%%%%%%%%%%%
\chapter*{Preface}
% A Quote all about Databases
\begin{chapquote}{Daniel Keys Moran}
``You can have data without information, but you cannot have information without data.''
\end{chapquote}

\noindent \BookAuthorName \\
\noindent \url{\BookAuthorURL}

%%%%%%%%%%%%%%%%%%%%%%%%%%%%%%%%%%%%
%%%%%~ NEW CHAPTER STARTS HERE %%%%%
%%%%%%%%%%%%%%%%%%%%%%%%%%%%%%%%%%%%

\chapter{Introduction to MySQL}

\section{What is MySQL?}

\textbf{MySQL} is an open-source relational database management system 
(RDBMS). An RDBMS is a type of database management system (DBMS) that 
stores data in a structured format, using rows and columns. It is a
software that is used to manage the creation, modification, and
maintenance of a database. MySQL is a popular choice for database
management in web applications and is used by many high-profile
websites, including Facebook, Twitter, and YouTube.

\section{MySQL Elements}

\subsection{Tables}

A \textbf{table} is a collection of related data stored in rows and
columns. Each row in a table represents a record, and each column
represents a field. Tables are the basic building blocks of a database
and are used to store and organize data in a structured format.

% Example of a table
\begin{table}[!h]
  \centering
  \begin{tabular}{|c|c|c|}
    \hline
    \textbf{id} & \textbf{name} & \textbf{age} \\
    \hline
    1 & Alice & 25 \\
    2 & Bob & 30 \\
    3 & Charlie & 35 \\
    \hline
  \end{tabular}
  \caption{Example of a table}
\end{table}

Table \ref{table:example} shows an example of a table with three
columns: \textbf{id}, \textbf{name}, and \textbf{age}. Each row in the
table represents a record with a unique \textbf{id} value, and each
column represents a field in the record.

\subsection{Columns}

A \textbf{column} is a vertical arrangement of data in a table. Each
column represents a field in the table and contains data of a specific
type. Columns are used to store different types of data, such as
numbers, text, dates, and more.

In the example table above, the columns are \textbf{id}, \textbf{name},
and \textbf{age}. The \textbf{id} column stores unique identifiers for
each record, the \textbf{name} column stores the names of the
individuals, and the \textbf{age} column stores the ages of the
individuals.

\subsection{Rows}

A \textbf{row} is a horizontal arrangement of data in a table. Each row
represents a record in the table and contains data for each column in
the table. Rows are used to store individual records in the table.

In the example table above, each row represents a record with data for
the \textbf{id}, \textbf{name}, and \textbf{age} columns. The first row
contains data for Alice, the second row contains data for Bob, and the
third row contains data for Charlie.

\subsection{Data Types}

A \textbf{data type} is a classification of data based on the type of
values it can hold. MySQL supports a wide range of data types, including
numeric, string, date, and time data types. Data types are used to
specify the type of data that can be stored in a column in a table.

\subsubsection{Numeric Data Types}

Numeric data types are used to store numeric values, such as integers,
decimals, and floating-point numbers. MySQL supports a variety of
numeric data types, including \textbf{INT}, \textbf{DECIMAL},
\textbf{FLOAT}, \textbf{DOUBLE}, \textbf{TINYINT}, \textbf{SMALLINT},
\textbf{MEDIUMINT}, and \textbf{BIGINT}.

\subsubsection{Character and String Data Types}

Character and string data types are used to sto andre text values, such as
names, addresses, and descriptions. MySQL supports a variety of
character and string data types, including \textbf{CHAR}, \textbf{VARCHAR},
and \textbf{TEXT}.

\subsubsection{Date and Time Data Types}

Date and time data types are used to store date and time values, such as
birthdates, appointment times, and event dates. MySQL supports a variety
of date and time data types, including \textbf{DATE}, \textbf{TIME},
\textbf{DATETIME}, and \textbf{TIMESTAMP}.

\subsubsection{Binary Data Types}

Binary data types are used to store binary data, such as images, audio
files, and video files. MySQL supports a variety of binary data types,
including \textbf{BLOB}, \textbf{MEDIUMBLOB}, and \textbf{LONGBLOB}.

\subsubsection{Boolean Data Type}

The \textbf{BOOLEAN} data type is used to store boolean values, such as
true or false. In MySQL, boolean values are represented as 1 for true and
0 for false.

\subsection{NULL Values}

A \textbf{NULL} value is a special value that represents the absence of
a value. NULL values are used to indicate that a column does not contain
any data. In MySQL, columns can be defined to allow NULL values, which
means that the column can contain NULL values in addition to other
values.

\subsection{Comments}

Comments are used to add explanatory notes to the SQL code. Comments are
ignored by the MySQL server and are used to document the code for
reference. Comments can be added to SQL code using the \textbf{--} or
\textbf{/* */} syntax.

\begin{lstlisting}[language=SQL, caption={Example of a comment in SQL code}, label={lst:comment}]
-- This is a single-line comment

/*
  This is a multi-line comment
  that spans multiple lines
*/

SELECT * FROM users; -- This is a comment at the end of a line
\end{lstlisting}

Code \ref{lst:comment} shows an example of comments in SQL code.
Comments can be added to SQL code using the \textbf{--} syntax for
single-line comments and the \textbf{/* */} syntax for multi-line
comments.

\chapter{Managing Databases}

\section{Creating a Database}

To create a new database in MySQL, you can use the \textbf{CREATE
DATABASE} statement followed by the name of the database. The following
example creates a new database named \textbf{database\_name}:

\begin{lstlisting}[language=SQL, caption={Creating a new database in MySQL}, label={lst:create-database}]
CREATE DATABASE database_name;
\end{lstlisting}

Code \ref{lst:create-database} shows an example of creating a new
database in MySQL using the \textbf{CREATE DATABASE} statement. The
statement creates a new database named \textbf{database\_name}. Once the
database is created, you can use it to store tables and data.

\section{Showing Databases}

To display a list of all databases in MySQL, you can use the \textbf{SHOW
DATABASES} statement. The following example shows a list of all databases
in MySQL:

\begin{lstlisting}[language=SQL, caption={Showing all databases in MySQL}, label={lst:show-databases}]
SHOW DATABASES;
\end{lstlisting}

Code \ref{lst:show-databases} shows an example of displaying a list of
all databases in MySQL using the \textbf{SHOW DATABASES} statement. The
statement lists all databases that are currently available in the MySQL
server.

\section{Selecting a Database}

To select a database in MySQL, you can use the \textbf{USE} statement
followed by the name of the database. The following example selects the
\textbf{database\_name} database:

\begin{lstlisting}[language=SQL, caption={Selecting a database in MySQL}, label={lst:select-database}]
USE database_name;
\end{lstlisting}

Code \ref{lst:select-database} shows an example of selecting a database
in MySQL using the \textbf{USE} statement. The statement selects the
\textbf{database\_name} database, which allows you to perform operations
on the tables and data in that database.

\section{Removing a Database}

To remove a database in MySQL, you can use the \textbf{DROP DATABASE}
statement followed by the name of the database. The following example
removes the \textbf{database\_name} database:

\begin{lstlisting}[language=SQL, caption={Removing a database in MySQL}, label={lst:drop-database}]
DROP DATABASE database_name;
\end{lstlisting}

Code \ref{lst:drop-database} shows an example of removing a database in
MySQL using the \textbf{DROP DATABASE} statement. The statement deletes
the \textbf{database\_name} database and all tables and data stored in
that database.

\noindent\textbf{Exercises}

\begin{lstlisting}[numbers=none]
1. Create a new database named "employees".
2. Display a list of all databases in MySQL.
3. Select the "employees" database.
4. Remove the "employees" database.
\end{lstlisting}

\chapter{Managing Tables}

\chapter{References}

\begin{enumerate}[label={\textbf{\Alph*.}}]
  \item \textbf{Books}
    \begin{itemize}
      \item 
    \end{itemize}
  \item \textbf{Other Sources}
    \begin{itemize}
      \item 
    \end{itemize}
\end{enumerate}

\end{document}
